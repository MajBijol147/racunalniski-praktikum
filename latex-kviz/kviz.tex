\documentclass[11pt]{article}

\usepackage[utf8]{inputenc}
\usepackage[T1]{fontenc}

\usepackage[slovene]{babel}
\usepackage{lmodern}
\usepackage{amsmath}
\usepackage{amsfonts}
\usepackage{amsthm}
\usepackage{mathtools}

\begin{document}

Poleg te datoteke ste dobili tudi datoteko z vprašanji \texttt{kviz-resitev.pdf}.
Te datoteke vam ni treba oblikovati, vanjo samo napišite odgovore.
Pri nekaterih nalogah je možnih več pravilnih odgovorov. Izberite vse.

\begin{enumerate}
    \item Napišite izraz za vsoto kadratov prvih n $n$ naravnih števil: \begin{equation}
      \sum_{i=1}^{n}i^2 = \frac{n(n+1)(2n+1)}{6}
    \end{equation}% tu napišite izraz

    \item Kako pravilno napišemo $\cos x$?
    \begin{table}[h]
      \centering
      \begin{tabular}{cc}
        (a) cos~x (\texttt{cos\textasciitilde x}) & (d)$\text{cos} ~ x$ (\texttt{\string \text{cos}\textasciitilde x})
      \end{tabular}
    \end{table} % tu napišite vse pravilne odgovore
    
    \item Kateri od teh simbolov je pravilno zapisan za oznako množice praznega prostora?
    \begin{table}[h]
      \centering
      \begin{tabular}{lcr}
         (a)$\phi$ , & (b)$\emptyset$, & (c) $\{\}$.
      \end{tabular}
    \end{table}
    % tu napišite vse pravilne odgovore
    
    \item Taylorjev razvoj funkcije $\sin x$ do tretjega člena:
    \[\sin x = x - \frac{x^3}{3!} + (x^5)\]

    \item Kako pravilno poravnamo matematični izraz na sredino?  % tu naštejte vse pravilne odgovore
    % (a)
      \[ e^{i \pi} + 1 = 0 \]
    % (b)
      \begin{center}
          \( e^{i \pi} + 1 = 0 \)
      \end{center}
    % (c)
      \begin{equation*}
          e^{i \pi} + 1 = 0
      \end{equation*}

    \item Kako prav delamo odstavke % tu naštejte vse pravilne odgovore
        \begin{enumerate}
          % (a)
            \item {
                Lorem ipsum dolor sit amet, consectetur adipiscing elit. Ut id viverra ligula. Phasellus vehicula lorem vitae luctus dignissim. 
                
                Sed ac justo commodo, fringilla urna ac, efficitur leo. Praesent dui odio, accumsan ac sapien nec, interdum volutpat est. 
            }
          % (b)
            \item {
                Lorem ipsum dolor sit amet, consectetur adipiscing elit. Ut id viverra ligula. Phasellus vehicula lorem vitae luctus dignissim. \\
                Sed ac justo commodo, fringilla urna ac, efficitur leo. Praesent dui odio, accumsan ac sapien nec, interdum volutpat est. 
            }
          % (c)
            \item {
                Lorem ipsum dolor sit amet, consectetur adipiscing elit. Ut id viverra ligula. Phasellus vehicula lorem vitae luctus dignissim. \par
                Sed ac justo commodo, fringilla urna ac, efficitur leo. Praesent dui odio, accumsan ac sapien nec, interdum volutpat est. 
            }
        \end{enumerate}
    
      \item Kateri način je pravilen za pisanje narekovajev in v katerm jeziku? % tu naštejte vse pravilne odgovore
      \begin{table}
        \centering
        \begin{tabular}{ll}
      \begin{enumerate}
          % (a)
            \item "Primer 1"
          % (b)
            \item ``Primer 2''
          % (c)
            \item ">Primer 3"<
          % (d)
            \item "`Primer 4"
        \end{enumerate}
      \end{tabular}
    \end{table}
\end{enumerate}

\end{document}
