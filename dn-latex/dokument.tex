\documentclass[11pt]{article}
\usepackage[a4paper, margin=2.5cm]{geometry}
\usepackage{amsthm}
\usepackage[slovene]{babel}
\usepackage[utf8]{inputenc}
\usepackage{makeidx}
\usepackage{amssymb}


\makeindex

\title{Brownovo gibanje}
\author{Matej Rojec}
% Brownovo gibanje
% Matej Rojec

{\theoremstyle{definition}
\newtheorem{definicija}{Definicija}}

{\theoremstyle{plain}
\newtheorem{izrek}{Izrek}}

\newcommand{\f}{\mathcal{F}}

\begin{document}
    \maketitle
    Brownovo gibanje (več v \cite{karatzas1991brownian}) je intuitivno slučajen proces,  % Sklic na knjigo
    ki predstavlja naključno gibanje delcev v mediju.
    \begin{figure}[h]
        \centering
        \includegraphics{PerrinPlot2.pdf}
        \caption{Reprodukcija slike iz Jean Baptiste Perrin, \emph{Mouvement brownien et réalité moléculaire}, Ann. de Chimie et de Physique (VIII) 18, 5-114, 1909}
    \end{figure} 

    % Slika: PerrinPlot2.pdf
    % Napis pod sliko: 
    % Reprodukcija slike iz Jean Baptiste Perrin, \emph{Mouvement brownien et réalité moléculaire}, Ann. de Chimie et de Physique (VIII) 18, 5-114, 1909

    % Začetek definicije
    \begin{definicija}
        Standardno Brownovo gibanje $\{B_t\}_{t \geq 0}$ je slučajen proces z naslednjimi lastnostmi: 
       \begin{enumerate}
        \item $B_0 = 0$.
        \item Prirastki $B_{t_n} - B_{t_{n-1}}, B_{t_{n-1}} - B_{t_{n-2}}, \ldots, B_2 - B_1, B_1 - B_0$ so neodvisne slučajne spremenljivke, za vsak $t_0 \leq t_1 \leq \cdots \leq t_{n-1} \leq t_n$.
        \item Za vsak $t \geq 0$ in $h > 0$ velja $B_{t+h} - B_t \sim \mathcal{N}(0, h)$.
        \item Funkcija $t \mapsto B_t$ je zvezna skoraj gotovo.
       \end{enumerate} 
    \end{definicija}
    % Konec definicije
    
    Preden zapišemo izrek, definirajmo še pojem časa ustavljanja.
    
    % Začetek definicije
    \begin{definicija}
        Slučajna spremenljivka $\tau$ na verjetnostnem prostoru $(\Omega, \f, P)$ z vrednostmi v $\mathbb{R}^+$
        je \emph{čas ustavljanja} glede na filtracijo $(\f_t)_{t\in T}$, če velja $\forall t \in T: \{\tau \leq t\} \in \f_t$.
    \end{definicija}
    % Konec definicije
    
    Zdaj lahko zapišemo izrek 1. % Sklic na izrek z oznako thm:stopped_brownian
    %~\cite{thm:stopped_brownian}
    % Začetek izreka
    \begin{izrek}
        Naj bo $\{B_t\}_{t \geq 0}$ (standardno) Brownovo gibanje, $\tau$ čas ustavljanja glede na 
        $(\f_t)_{t\geq 0}$ in naj velja $P[\tau < \infty] = 1$.
        Potem je tudi proces:
        \[
        \hat{B} := \{B_{T+t} - B_T \mid t \geq 0\}
        \]
        (standardno) Brownovo gibanje in neodvisen od $\f_T$.
    \end{izrek}
    % Konec izreka
    
    \bibliographystyle{plain}
    \bibliography{knjiga}
    \printindex
\end{document}